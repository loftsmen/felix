\documentclass[oneside]{book}
\usepackage{color}
\definecolor{cxxbg}{rgb}{0.95,0.85,0.95}
\definecolor{felixbg}{rgb}{0.95,0.95,0.95}
\definecolor{felixlibbg}{rgb}{0.95,0.95,0.85}
\definecolor{emphcolor}{rgb}{0.5,0.0,0.0}
\newcommand{\empha}{\bf\color{emphcolor}}
\usepackage{parskip}
\usepackage{framed}
\usepackage[newfloat=true]{minted}
\usepackage{amsmath}
\usepackage{amssymb}
\usepackage{amscd}
\usepackage{imakeidx}
\usepackage[chapter]{tocbibind}
\usepackage{tikz}
\usetikzlibrary{shapes,shadows,arrows}
\makeindex[title=General Index]
\makeindex[name=codeindex,title=Code Index]
\usemintedstyle{friendly}
\setminted{bgcolor=felixlibbg,xleftmargin=20pt}
\usepackage{hyperref}
\hypersetup{pdftex,colorlinks=true,allcolors=blue}
\newcommand*{\fullref}[1]{\hyperref[{#1}]{\autoref*{#1} \nameref*{#1}}}
\usepackage{hypcap}
\usepackage{caption}
\DeclareMathOperator{\quot}{div}
\DeclareMathOperator{\rmd}{rmd}
\title{Felix Dependency Processing}
\author{John Skaller}
\begin{document}
\maketitle
\tableofcontents
\chapter{Introduction}
Felix has several different subsystems and features which can loosely
be described as dependency processing.

\chapter{Symbol dependencies}

In Felix when you use an identifier, you create a dependency on
that symbol's definition. You top level code is wrapped in a dummy
procedure \verb%_init_% which is considered the root symbol,
and then any symbols are also considered used. The consideration
is recursive, so Felix finds the transitive closure of the usage
dependency and deletes all the unused symbols, so as to reduce
the total amount of code it eventually generates. This is, of course,
a kind of garbage collection.

There is a vital question here: exactly what does it mean for
a symbol to be used?

With a special exception, it means that the symbol appears in
an expression, is a parameter, is the type of any used variable,
the type of a function, etc. Note, taking the address of a 
variable is considered a use (even if the address is not used
to access the variable dynamically).

The special exception is this: if a variable is assigned to or
initialised by is not used in an expression, it is not considered
to be used. Instead, the assignment or initialisation statement
is deleted. In turn, any variables which are only used in that
assignment now become unused, since the assignment was deleted.
After all the assigment is useless since the variable assigned
to is not used.

Felix guarrantees any unused variable will be deleted, along
with its initialisation, however note that parameters are
considered to be used, because deleting them would change
the type of the function that they are parameters of.

This has no impact, except to improve performance, if the
initialiser or RHS of an assignment is side effect free.

However if the initialiser or assignment RHS has a side effect
which is allowed if it contains the application of a generator,
that side effect will be lost.

This is not a design fault, but a deliberate design choice.
The specific use case which guides this choice is the assignment
of a real time alarm clock to a global variable. If the clock
is not used, the variable is removed, the assignment deleted,
and the construction of the clock bypassed. This is important
because alarm clocks use the asynchronous I/O subsystem,
which is loaded on demand if you use dynamic linkage.
What's more, the subsystem starts a pre-emptive thread which
monitors the operating systems event queue, which is a lot
of overhead for a clock you do not use.

With the elision assured, it is safe to construct variables
refering to expensive resources, knowing the construction
will be elided unless you use the resource via the variable.

\section{C++ Dependencies}
Felix is a C++ code generator which is specially designed to allow
the programmer to leverage existing C and C++ code, and to blend
in C and C++ code right into a Felix program.

Consider for example you want a function \verb%myffun% to perform a 
calculation which you want to write in C++. The first thing you need to do
is create a binding to the function:

\begin{minted}{felix}
fun myffun : double -> double = "::std::sin (mycfun($1))";
\end{minted}

Unfortunately, this is probably not going to work because it
needs the C++ math function \verb%sin% and to get use that 
function we have to include the C++ header \verb%cmath%.
Lets do that first:


\begin{minted}{felix}
fun myffun : double -> double = "::std::sin (mycfun($1))"
  requires header '#include <cmath>';
\end{minted}

What this does is that, if the Felix function \verb%myffun%
is used, then, Felix will automatically include the line

\begin{minted}{c}
#include <cmath>
\end{minted}

in the C++ program it generates, in particular in the file
with the basename of the Felix filename and extension \verb%.hpp%.
This file is normally written into the cache prior to use during
the C++ compilation phase.

So now the \verb%sin% bit works, but what about the C function
you were going to write. Lets write it:

\begin{minted}{felix}
body mycfun_def =
"""
double mycfun (double x) { return x + 1.0; }
""";
\end{minted}

Here we have written the C function \verb%mycfun% inside a string.
We used triple quotes to support multiline strings.
We also tagged the string \verb%mycfun_def% and told Felix it
has to go in the body file it generates, which as a \verb%.cpp%
extension.

If you put this before the binding of \verb%myfffun% however,
the program still won't run, because the C function is not
included in the output!

The reason of course is that you did not create a dependency
on it, and the cure is to do so:

\begin{minted}{felix}
fun myffun : double -> double = "::std::sin (mycfun($1))"
  requires header '#include <cmath>', mycfun_def;
\end{minted}



\clearpage
\phantomsection
\indexprologue{Listing Index}
\listoflistings
%
\clearpage
\phantomsection
\printindex[codeindex] 
%
\clearpage
\phantomsection
\printindex
%
\end{document}
