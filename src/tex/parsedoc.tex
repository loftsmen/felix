\documentclass[oneside]{book}
\usepackage{xcolor}
\definecolor{bg}{rgb}{0.95,0.95,0.95}
\definecolor{emphcolor}{rgb}{0.5,0.0,0.0}
\newcommand{\empha}{\bf\color{emphcolor}}
\usepackage{parskip}
\usepackage{minted}
\usepackage{caption}
\usepackage{amsmath}
\usepackage{amssymb}
\usepackage{amscd}
\usemintedstyle{friendly}
\setminted{bgcolor=bg,xleftmargin=20pt}
\usepackage{hyperref}
\hypersetup{pdftex,colorlinks=true,allcolors=blue}
\usepackage{hypcap}
\title{Designing a Parser}
\author{John Skaller}
\begin{document}
\maketitle
\tableofcontents
\chapter{Introduction}
Our aim is to produce an {\em executable, scannerless} parser. By executable we mean the grammar
to parse is constructed at run time, and the parser uses this grammar to parse strings. By scannerless
we mean that there is no lexer, instead all tokens can be considered regular expressions.

Our parser will be a Generalised LR (GLR) shift/reduce parser. This means multiple parse threads
exist at one time but proceed in lockstep.

Because we intend to parse program input, it suffices to load the entire input into memory
as an array, since programmers produce relatively short source files. The current position
in the input can be indicated by a pointer or offset from the start of the text.

We first need a procedure \verb%skipwhite% which skips white space and comments. It is applied
initially to the input buffer to sit the current position at the start of significant
lexeme or the end of the input. If we're not at the end, we can locate a lexeme, and then apply
the \verb%skipwhite% procedure again.

Each parse thread is represented by an object with three components.
The first is the input string and the current location therein.
The second is the current location in the grammar, represented by the control stack.
The third is a stack of as yet unreduced symbols.

Our grammar starts of a set of nonterminals of which one is designated the start symbol.
Each nonterminal is mapped to a set of labelled productions, representing the alternatives
for that nonterminal. A production is a sequence of symbols, either terminals or nonterminals.
Nonterminals are regular expressions.

In order that scanning for lexemes make progress, if the longest match of a regular
expression for a terminal on a position in the input string is empty, the match is
considered to fail. This could be assured by requiring the regular expressions to 
not match the empty string, however simply failing a longest match which makes
no progress is an alternatives.

When we're matching a nonterminal, we must find all the nonterminals which could
occur at this point in the parse. We can do this by recursively examining all of the
alternative productions for that nonterminal. We consider the first symbol in each
production. If it is a nonterminal we recurse, if it is a terminal we add it to
the set of possible next nonterminals.

In order that this algorithm terminate we will use two constraints on the grammar.
First, the grammar does not contain any epsilon (empty) productions.
Second, it must not be left recursive, as this would cause an infinite loop.
With this constraints, any recursive descent searching for terminals must bottom out.

Now, we need to do more than just return the set of regular expressions.
We need to know the path in the grammar from the current position to the
associated terminal. Thus the set of objects we return will be a set
of pairs consisting of the terminal regular expression and the path.
The path is simply a seqence of production labels.

Now, we match each of the returned regular expressions against the input
in turn, keeping on the longest matches. Each of the final set of objects
represents a match of exactly the same lexeme so the lexeme can be extracted,
the input position advanced, and the white space skipper applied to ready
the input for the next matching operation.

If the set is empty, the parse fails. Otherwise we push the terminal
and recognised lexeme onto the symbol stack. Now we need to spawn
a set of new threads, each being the update of the parse thread
but with one of the returned paths pushed onto the control stack.

Now, for each thread, if the path is at the end of a production
of length $n$, we pop $n$ symbols off the data stack and replace
them with a node labelled by the production label; and whose
children are the symbols popped off the stack. We also pop the
top entry off the control stack. Now we repeat this process
until we reach the start symbol or a non-final location in
some production. If we reached the start symbol 
then the stack will contain a single node representing
the parse of the input up to that point in the input.
If we seek to parse the whole string then if the input is exhausted
add the stacked node to the set of of final parses.
Otherwise add the parse and the input location so we know which prefix
of the input this parse tree represents.

The process above forks the parser on each alternative, and drops
threads that fails to match the input. If the grammar is ambiguous,
then the final result consists of a set of parse trees, possibly empty.
The parse terminates when no threads are left.



\appendix
\backmatter
\end{document}
